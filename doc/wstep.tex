\section{Cel i zakres projektu}
\label{sec:cel}

Celem projektu było zaimplementowanie modułu obsługi myszy za pomocą protokołu
PS/2 w układzie FPGA przy użyciu języka VHDL.
\vspace{1em}

Finalną wersją projektu jest prosta aplikacja graficzna typu \texttt{Paint}, pozwalająca
na rysowanie punktow za pomocą myszy na ekranie monitora \texttt{VGA}, przechowująca 
mapę obrazu w pamięci \texttt{RAM}.
\vspace{1em}

Projekt został wykonany na układzie Xilinx Spartan 3E Starter (XC3S500E
w~obudowie FG320). Dodatkowym sprzętem wykorzystanym w projekcie jest monitor \texttt{CRT}
ze złaczem \texttt{D-SUB} oraz mysz podłączana za pmocą portu \texttt{PS/2}.

\section{Teoria problemu}
\label{sec:teoria}
\subsection{Protokół PS/2}
Protokół PS/2 oparty jest na dwukierunkowej transmisji szeregowej.
Wykorzystywane są dwie linie –- linia danych oraz linia zegara. Do każdej z nich
podłączony jest rezystor podciągający,
\vspace{1em}

Podczas odbioru danych od myszy to mysz generuje przebieg zegarowy. Linia danych
znajduje się w stanie stabilnym podczas opadającego zbocza sygnału zegarowego.
Ramka składa się z bitu startu (\texttt{0}), ośmiu bitów danych (od najmłodszego
do najstarszego), bitu parzystości oraz bitu stopu (\texttt{1}).
\vspace{1em}

Aby nadać bajt urządzenie sterujące transmisją musi wymusić stan \texttt{0} na
linii zegara przez czas $100 \mu$s. Po upływie tego czasu mysz zaczyna generować
sygnał zegarowy. Następnie na linię danych podawane są bity ramki w formacie
identycznym jak w przypadku odbioru. Sygnał na linii danych powinien być
stabilny podczas opadającego zbocza zegarowego. Do implementacji transmisji
dwukierunkowej należy użyć bufora trójstanowego.
\vspace{1em}

Mysz zaczyna pracę w trybie, w którym nie wysyła żadnych danych. Należy ją
zainicjalizować wysyłając bajt \texttt{0xF4}. Mysz odpowiada bajtem
potwierdzenia oraz zaczyna wysyłać pakiet danych, który składa się z trzech
bajtów. Zawiera on informacje o stanie przycisków oraz przesunięciu w
osi X oraz Y. Informacje o przesunięciach zapisane są jako liczba binarna U2.

\subsection{Wyświetlacz VGA}
W projekcie wykorzystany został również wyświetlacz CRT w trybie
640$\times$480@60Hz i pozwalał na wyświetlanie do ośmiu kolorów, jednak w
projekcie wykorzystane zostały tylko kolory biały (zapalony piksel) oraz czarny
(zgaszony piksel). Aby wyświetlić obraz należało wygenerować przebieg zegarowy o
częstotliwości 25MHz oraz sterować sygnałami synchronizacji poziomej, pionowej
oraz bitami odpowiadającymi za wyświetlany kolor.


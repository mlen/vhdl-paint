\section{Podsumowanie}
\label{sec:podsumowanie}

Mimo pozytywnego efektu końcowego naszego projektu, uwidocznionego na
zdjęciu~\ref{fig:dziala}, można w nim wprowadzić szereg usprawnień.  Pierwszym
i najważniejszym czynnikiem jest zablokowanie ekranu, przez co kursor myszy nie
będzie zapętlał się, tylko zatrzymywał na krawędzi ekranu ograniczonej rozmiarem
pamięci \texttt{RAM}.

\vspace{1em}
W projekcie wykorzystanie pamięci przy ekranie 256$\times$256 pikseli wynosi
zaledwie 20\%. Daje nam to możliwość stworzenia większego ekranu 512$\times$512,
lub wykorzystania 8 kolorów.

\vspace{1em}
Kolejnym usprawnieniem powinna być możliwość czyszczenia całego ramu, a nie jak
obecnie wymazywanie poszczególnych pikseli.

\vspace{1em}
Ostatecznie, nazwy modułów w kodzie powinny zostać zmienione, na bardziej
czytelne dla użytkownika kodu, lecz z powodu nieprzewidywalnych zachowań
środowiska programistycznego \texttt{Xilinx}, nie zdecydowaliśmy się na ten
krok.

